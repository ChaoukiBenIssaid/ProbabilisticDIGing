\documentclass[12pt,draftclsnofoot,onecolumn]{IEEEtran}
%\documentclass[journal,10pt]{IEEEtran}

\IEEEoverridecommandlockouts
\usepackage{algorithm,algorithmic,multirow,hhline}
\usepackage{hyperref}
\newtheorem{remark}{Remark}
\newtheorem{theorem}{Theorem}
\newtheorem{proposition}{Proposition}
\newtheorem{lemma}{Lemma}
\newtheorem{definition}{Definition}
\newtheorem{proof}{Proof}
\usepackage{tabularx}
\usepackage{array}
\usepackage{booktabs}
\usepackage{url}
\usepackage[noadjust]{cite}

\def\ie{\emph{i.e.\  }}
\def\eg{\emph{e.g.\  }}
\providecommand{\Ex}[1]{\mathbb{E}\left[#1\right]}
\providecommand{\abs}[1]{\left|#1\right|}
\providecommand{\norm}[1]{\left\|#1\right\|}
\providecommand{\ip}[1]{\boldsymbol{\langle}#1\boldsymbol{\rangle}}
\newcommand{\tred}[1]{{\textcolor{red}{#1}}}
\newcommand{\tblue}[1]{{\textcolor{blue}{#1}}}
\def\sm{\small}
\def\nm{\normalsize}
\usepackage{bm}

\usepackage{amsmath,graphicx}
\usepackage{placeins}
\usepackage{amsmath,epsfig}
\usepackage{epstopdf}
\usepackage{amssymb}
\providecommand{\tabularnewline}{\\}

\usepackage{amssymb}
\usepackage{bbm}
\usepackage{caption}
\usepackage{comment}
%\usepackage{subfigure}
\usepackage{subcaption}
\usepackage{pstricks}
\usepackage{subfloat}
\usepackage{setspace}
\usepackage[left=0.58in, right=0.58in, top=0.71in]{geometry}
\newenvironment{thisnote}{\par\color{blue}}{\par}
\usepackage{xpatch}
\xpatchcmd\algorithmic
  {\newcommand{\IF}}
  {%
    \newcommand\SCOPE{\begin{ALC@g}}%
    \newcommand\ENDSCOPE{\end{ALC@g}}%
    \newcommand{\IF}%
  }
  {}{\fail}


	% \usepackage{stfloats} % bottom floating equations
\usepackage{dblfloatfix}



\begin{document}

\title{Probabilistic DIGing}

%\author{Author 1, Author 2, Author 3}
%\author{Chaouki Ben Issaid, Anis Elgabli, Mehdi Bennis} %\thanks{C. Ben Issaid, A. Elgabli and M. Bennis are with the Centre of Wireless Communications, University of Oulu, 90014 Oulu, Finland (email: \{chaouki.benissaid, anis.elgabli, mehdi.bennis\}@oulu.fi).}} 


\maketitle

\section{Brief Description}
The optimization problem is given by
\begin{align}
\textbf{(P1)} ~~ &\bm{\Theta}^* := \arg\min_{\bm{\Theta}} \sum_{n=1}^N f_n(\bm{\Theta}),
\end{align}
where $\bm{\Theta} \in  \mathbb{R}^{d \times 1}$ is the model parameter and $f_n: \mathbb{R}^d \rightarrow \mathbb{R}$ is a local function composed of data stored at worker $n$.

\subsection{Linear Regression}
\subsubsection{Loss Function}
In this case, the local cost function at worker $n$ is explicitly given by
\begin{align}
f_n(\bm{\theta}) = \frac{1}{2} \| \bm{X}_n \bm{\theta} - \bm{y}_n\|^2,
\end{align}
where $\bm{X}_n \in \mathbb{R}^{s \times d}$ and $\bm{y}_n \in \mathbb{R}^{s \times 1}$ are private for each worker $n \in \mathcal{V}$ where $s$ represents the size of the data at each worker.
\subsubsection{Datasets}
In this task, we will consider the following datasets (see Table 2 of \url{http://cacr.uwaterloo.ca/techreports/2019/cacr2019-05.pdf}):
\begin{itemize}
\item Boston: \url{https://github.com/benchopt/benchmark_ols/tree/master/datasets}
\item Wine Quality: \url{https://archive.ics.uci.edu/ml/datasets/wine+quality}
\end{itemize}
\subsection{Logistic Regression}
\subsubsection{Loss Function}
In this subsection, we consider the $L_2$-regularized binary logistic regression task. We assume that each worker $n$ owns a data matrix $\bm{X}_n = (\bm{x}_{n,1}, \dots, \bm{x}_{n,s})^T \in \mathbb{R}^{s \times d}$ along with the corresponding labels vector $\bm{y}_n = (y_{n,1}, \dots, y_{n,s}) \in \{-1, 1\}^{s}$. The local cost function for worker $n$ is then given by 
\begin{align}
f_n(\bm{\theta}) = \frac{1}{s} \sum_{j=1}^s \log\left( 1 + \exp\left(- y_{n,j} \bm{x}_{n,j}^T \bm{\theta} \right)\right) + \frac{\lambda}{2} \|\bm{\theta}\|_2^2,
\end{align}
where $\lambda$ is the regularization parameter.
\subsubsection{Datasets}
In this task, we will consider the following datasets:
\begin{itemize}
\item a1a:  \url{https://github.com/konstmish/opt_methods/tree/master/datasets}
\item mushrooms: \url{https://github.com/konstmish/opt_methods/tree/master/datasets}
\item w8a: \url{https://github.com/konstmish/opt_methods/tree/master/datasets}
\item madelon: \url{https://github.com/benchopt/benchmark_logreg_l2/tree/master/datasets}
\item covtype: \url{https://github.com/benchopt/benchmark_logreg_l2/tree/master/datasets}
\end{itemize}

\begin{table*}[h]
\centering
\begin{tabular}{|l|l|c|c|c|}
\hline
\textbf{Dataset} & \textbf{Task} & \textbf{Model Size ($d$)} & \textbf{Number of Instances} & \textbf{Number of Workers ($N$)} \\ \hline \hline
  & linear regression      & $50$  & $1200$ & $ $ \\ \hline
 & linear regression           &  &  & \\ \hline
 & logistic regression  &  & & \\ \hline
 & logistic regression       &  & & \\ \hline
\end{tabular}
\caption{List of datasets used in the numerical experiments.}
\label{table} 
\end{table*}

\subsection{Websites for datasets}
\begin{itemize}
\item \url{https://www.openml.org/}
\item \url{https://archive.ics.uci.edu/ml/datasets.php}
\end{itemize}

\subsection{Tasks}
\begin{itemize}
\item Dataset preparation: download at least 4 datasets for each task (regression/classification) in the format $X$ and $y$ (as .npy files if possible, otherwise any format you are comfortable with).
\item Strongly convex case
\begin{itemize}
\item Start simple: implement DGD and DIGing for the linear regression + logistic regression cases.
\item Implement the probabilistic DIGing and experiment with different choice of the probability $p$.
\begin{itemize}
\item[(i)] $p^k = \frac{a}{a + k}, ~~ a > 0$.
\item[(ii)] $p^k = \exp\left(-\frac{k}{T}\right), ~~ T > 0$.
\end{itemize}
\item Plots (need to think more about it): 
\begin{itemize}
\item[(i)] train/test loss (accuracy) or residuals vs. number of iterations.
\item[(ii)] train/test loss (accuracy) or residuals vs. cumulative communication cost. How to define the cumulative communication cost in our context?
\begin{align}
C_T^k = \sum_{n=1}^N d_n c^k
\end{align}
where $c^k$ is the size of the variables exchanged at iteration $k$.
\end{itemize}
\item Proof attempt.
\end{itemize}
\item Stochastic version of above (double-check the literature).
\item Non-convex case.
\end{itemize}

%\input{Introduction}
%\input{Notations}
%\input{formulation}
%\input{algorithm}
%\input{analysis}
%\input{evaluation}
%\section{Conclusions}\label{SecConc}

%\vspace{-5.5cm}
%\bibliographystyle{IEEEtran}
%\bibliography{references}

\end{document}